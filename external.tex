\documentclass{article}
\usepackage{graphicx}
\usepackage{subcaption}
\usepackage{amsmath}

\begin{document}
	\begin{figure}[h]
		\centering
		\begin{subfigure}[b]
			{0.45\textwidth}
			\centering
			\includegraphics[width=\linewidth]{logo.png}
			\caption{Description of Image 1.}
			\label{fig:image 1}
			
			
		\end{subfigure}
		\hfill
		\begin{subfigure}[b]
			{0.45\textwidth}
			\centering
			\includegraphics[width=\linewidth]{logo.png}
			\caption{Description of Image 2.}
			\label{fig:image 2}
		\end{subfigure}
		\label{fig:sidebyside}
	\end{figure}
  length of the bottom base is l 2 , the width of the bottom base is w 2 ,and the height is h is given by: \\

\[V = \frac{h}{3}(l_1w_1 + l_2w_2 + \sqrt{l_1w_1l_2w_2})\]
 (b) The surface area S of a frustum of a cone with the radii of the two circular bases being r 1 and r 2 , and the slant height being s is given by:
	
	\[S = \pi(r_1 + r_2)s + \pi r_1^2 + \pi r_2^2\]
	
	
\end{document}